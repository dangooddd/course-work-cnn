\section*{ВВЕДЕНИЕ}
\addcontentsline{toc}{section}{\protect ВВЕДЕНИЕ}

Классификация изображений является одной из основных задач современного компьютерного зрения.
С появлением вычислительных устройств человечество начало искать программные решения для задач которые ранее выполнялись исключительно человеком.
Примерами таких задач являются распознавание символов, медицинская диагностика и анализ рентгеновских снимков, сортировка брака на производстве, распознавание лиц и многие другие.

С появлением глубокого обучения многие перечисленные проблемы к текущему моменту были решены.
В частности, наиболее эффективными оказались так называемые "сверточные" нейронные сети, также известные как CNN (Convolutional Neural Network), речь о которых пойдет в данной работе.
Подробнее о понятии свертки рассматривается в теоретическом разделе.

Идея сверточных сетей имеет как математические, так и биологические основания.
Основная идея CNN пришла из нейрофизиологии.
Дэвидом Хьюбелом было обнаружено, что нейроны первичной зрительной коры реагируют на локальные шаблоны или признаки (например, линии под определенным углом), а нейроны более глубоких слоев на основе этих признаков выделяют все более сложные структуры \cite{dlbook}.
Этим также обосновывается использование в CNN не только сверточных, но и других слоев, которые обобщают работу сверточных.

Идея того, как именно в сверточных сетях организована "локальность", пришла из математики.
До появления глубокого обучения при работе с изображениями уже использовали так называемые фильтры -- матрицы небольшого размера, как бы "скользящие" по изображению.
С помощью правильно подобранного фильтра, например, можно выделять границы различных объектов или иные свойства.
Сверточные сети подбирают эти фильтры вместо человека, используя математический аппарат глубокого обучения. Так называемое "скольжение" фильтра по изображению можно описать математической операцией свертки, от которой CNN и получили свое называние.

Сверточные нейронные сети решили проблему огромного числа параметров, которые используются полносвязными нейронными сетями.
При высоком разрешении входно изображения, например $512 \times 512$ (по современным меркам даже такое разрешение уже не является большим),
требует для каждого нейрона первого слоя иметь $512 \times 512 = 262144$ параметров, что невероятно много даже для современных компьютеров.

Несмотря на появление новых алгоритмов глубокого обучения, с помощью которых также решается задача классификации, сверточные сети все еще остаются актуальными.
Помимо малого числа параметров по сравнению с полносвязными нейронными сетями, сверточные сети лучше поддаются анализу и интерпретации работы модели.
На текущий момент разработаны методы, позволяющие визуализировать выделяемые сетями признаки \cite{gradcam}.

Сама же задача классификации изображений остается актуальной и по сей день.
Сегодня ее используют при создании автономных автомобилей, в социальных сетях и рекомендательных системах, робототехнике.

\subsection*{Постановка задачи}
Целью работы является изучение методов разработки, обучения и тестирования сверточной нейронной сети, решающей задачу классификации изображений из датасета \detokenize{https://www.kaggle.com/datasets/alessiocorrado99/animals10}.