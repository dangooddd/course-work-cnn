\section*{ЗАКЛЮЧЕНИЕ}
\addcontentsline{toc}{section}{\protect ЗАКЛЮЧЕНИЕ}

В результате работы были изучены архитектуры моделей сверточных нейронных сетей AlexNet и VGG, общие методы глубокого обучения, метод интерпретирования сверточных моделей Grad-CAM.
Были предложены модификации указанных архитектур, решающие поставленную в работе задачу по классификации изображений.

В рамках работы выяснились недостатки AlexNet по сравнению с VGG13.
C помощью Grad-CAM выяснилось, что часть классов модель AlexNet определяет по окружению, а не по животному.
CVGG13 при этом лишена таких недостатков и рекомендуется к использованию в задаче классификации представленых в датасете 10 классов.

Полученная моделями точность (>80\%) является более чем удовлетворительной для малого обучающего набора. Таким образом, была достигнута цель работы.