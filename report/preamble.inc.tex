\usepackage{fontspec}
\setmainfont{Times New Roman}[Ligatures=TeX]
\newfontfamily\cyrillicfont{Times New Roman}[Ligatures=TeX]
\setmonofont{JetBrains Mono}
\newfontfamily\cyrillicfonttt{JetBrains Mono}

\usepackage{polyglossia}
\setmainlanguage{russian}
\setotherlanguage{english}

\usepackage{geometry}
\geometry{
    a4paper,
    left=30mm,
    right=15mm,
    top=20mm,
    bottom=20mm,
}

% Межстрочный интервал и абзацы
\usepackage{setspace}
\onehalfspacing % Полуторный интервал
\usepackage{indentfirst}
\setlength{\parindent}{1.25cm} % Отступ первой строки 1.25 см

\usepackage{fix-cm} % Корректное масштабирование шрифтов
\renewcommand{\normalsize}{\fontsize{14pt}{21pt}\selectfont}

% Математика
\usepackage{amsmath, amssymb, amsthm}
\usepackage{mathtools}
\usepackage{tikz}
\usetikzlibrary{shapes, arrows, positioning, fit}

% Теоремы
\theoremstyle{plain}
\newtheorem{theorem}{Теорема}[section]
\newtheorem{lemma}{Лемма}[section]
\newtheorem{corollary}{Следствие}[section]
\theoremstyle{definition}
\newtheorem{definition}{Определение}[section]

% Списки
\usepackage{enumitem}
\setlist{nosep, leftmargin=2cm} % Отступ 2 см, без лишних промежутков

% Таблицы и графика
\usepackage{array, booktabs, longtable, multirow}
\usepackage{graphicx}
\usepackage[skip=10pt, labelfont=bf]{caption}
\captionsetup{labelsep=endash} % Подписи вида "Рис. 1 – Описание"
\renewcommand{\captionlabelfont}{\bfseries\fontsize{12pt}{18pt}\selectfont}
\renewcommand{\captionfont}{\fontsize{12pt}{18pt}\selectfont}

% Ссылки и оглавление
\usepackage[unicode, hidelinks]{hyperref}
\usepackage{nameref}
\usepackage{tocloft}
\renewcommand{\cftsecleader}{\cftdotfill{\cftdotsep}} % Точки в оглавлении
\renewcommand{\cfttoctitlefont}{\bfseries\fontsize{16pt}{24pt}\selectfont}
\renewcommand{\cftsecfont}{\normalfont}
\renewcommand{\cftsubsecfont}{\normalfont}
\renewcommand{\cftsubsubsecfont}{\normalfont}
\renewcommand{\cftsecpagefont}{\normalfont}
\renewcommand{\cftsubsecpagefont}{\normalfont}
\renewcommand{\cftsubsubsecpagefont}{\normalfont}
\renewcommand{\cftsecaftersnum}{.}

\addto\captionsrussian{
    \renewcommand{\contentsname}{СОДЕРЖАНИЕ}
    \renewcommand{\figurename}{Рисунок}
    \renewcommand{\lstlistingname}{Листинг}
    \renewcommand\refname{\hfill СПИСОК ИСПОЛЬЗОВАННЫХ ИСТОЧНИКОВ \hfill}
}

% Заголовки
\usepackage{titlesec}
\titleformat{\section}{\bfseries\fontsize{16pt}{24pt}\selectfont\centering}{\thesection.}{0.5em}{}
\titleformat{\subsection}{\bfseries\fontsize{16pt}{24pt}\selectfont\centering}{\thesubsection}{0.5em}{}

% Библиография
\makeatletter
\renewcommand\@biblabel[1]{#1.}
\makeatother
\usepackage{hyperref}
\usepackage{url}

% Листинг
\usepackage{xcolor}
\usepackage{listings}

\definecolor{codegreen}{rgb}{0,0.6,0}
\definecolor{codeblue}{rgb}{0,0,0.8}
\definecolor{codered}{rgb}{0.8,0,0}

\lstset{
    % Базовое оформление---
    basicstyle=\ttfamily\small,
    backgroundcolor=\color{white},
    frame=lines, % Рамка сверху и снизу
    rulecolor=\color{black},
    framerule=0.5pt,
    % Переносы---
    breaklines=true,
    breakatwhitespace=true,
    breakindent=20pt,
    % Нумерация
    numbers=left, % Номера строк слева
    numberstyle=\tiny\color{gray},
    stepnumber=1,
    numbersep=5pt,
    % Поддержка кириллицы (типа)
    inputencoding=utf8,
    extendedchars=true,
    literate={},
    tabsize=4, % Размер табуляции
    showspaces=false, % Не показывать пробелы
    showstringspaces=false,
    showtabs=false,
    captionpos=t, % Позиция заголовка (top)
    belowcaptionskip=5pt,
    % Цвета для языков---
    commentstyle=\color{codegreen},
    keywordstyle=\color{codeblue},
    stringstyle=\color{codered},
    identifierstyle=\color{black},
}

\usepackage{enumitem}
\setlist[itemize]{
    leftmargin=*,
    itemindent=-1em,
    labelsep=0.5em,
    label={---}
}